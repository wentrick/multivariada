% Options for packages loaded elsewhere
\PassOptionsToPackage{unicode}{hyperref}
\PassOptionsToPackage{hyphens}{url}
%
\documentclass[
]{article}
\usepackage{amsmath,amssymb}
\usepackage{lmodern}
\usepackage{iftex}
\ifPDFTeX
  \usepackage[T1]{fontenc}
  \usepackage[utf8]{inputenc}
  \usepackage{textcomp} % provide euro and other symbols
\else % if luatex or xetex
  \usepackage{unicode-math}
  \defaultfontfeatures{Scale=MatchLowercase}
  \defaultfontfeatures[\rmfamily]{Ligatures=TeX,Scale=1}
\fi
% Use upquote if available, for straight quotes in verbatim environments
\IfFileExists{upquote.sty}{\usepackage{upquote}}{}
\IfFileExists{microtype.sty}{% use microtype if available
  \usepackage[]{microtype}
  \UseMicrotypeSet[protrusion]{basicmath} % disable protrusion for tt fonts
}{}
\makeatletter
\@ifundefined{KOMAClassName}{% if non-KOMA class
  \IfFileExists{parskip.sty}{%
    \usepackage{parskip}
  }{% else
    \setlength{\parindent}{0pt}
    \setlength{\parskip}{6pt plus 2pt minus 1pt}}
}{% if KOMA class
  \KOMAoptions{parskip=half}}
\makeatother
\usepackage{xcolor}
\usepackage[margin=1in]{geometry}
\usepackage{color}
\usepackage{fancyvrb}
\newcommand{\VerbBar}{|}
\newcommand{\VERB}{\Verb[commandchars=\\\{\}]}
\DefineVerbatimEnvironment{Highlighting}{Verbatim}{commandchars=\\\{\}}
% Add ',fontsize=\small' for more characters per line
\usepackage{framed}
\definecolor{shadecolor}{RGB}{248,248,248}
\newenvironment{Shaded}{\begin{snugshade}}{\end{snugshade}}
\newcommand{\AlertTok}[1]{\textcolor[rgb]{0.94,0.16,0.16}{#1}}
\newcommand{\AnnotationTok}[1]{\textcolor[rgb]{0.56,0.35,0.01}{\textbf{\textit{#1}}}}
\newcommand{\AttributeTok}[1]{\textcolor[rgb]{0.77,0.63,0.00}{#1}}
\newcommand{\BaseNTok}[1]{\textcolor[rgb]{0.00,0.00,0.81}{#1}}
\newcommand{\BuiltInTok}[1]{#1}
\newcommand{\CharTok}[1]{\textcolor[rgb]{0.31,0.60,0.02}{#1}}
\newcommand{\CommentTok}[1]{\textcolor[rgb]{0.56,0.35,0.01}{\textit{#1}}}
\newcommand{\CommentVarTok}[1]{\textcolor[rgb]{0.56,0.35,0.01}{\textbf{\textit{#1}}}}
\newcommand{\ConstantTok}[1]{\textcolor[rgb]{0.00,0.00,0.00}{#1}}
\newcommand{\ControlFlowTok}[1]{\textcolor[rgb]{0.13,0.29,0.53}{\textbf{#1}}}
\newcommand{\DataTypeTok}[1]{\textcolor[rgb]{0.13,0.29,0.53}{#1}}
\newcommand{\DecValTok}[1]{\textcolor[rgb]{0.00,0.00,0.81}{#1}}
\newcommand{\DocumentationTok}[1]{\textcolor[rgb]{0.56,0.35,0.01}{\textbf{\textit{#1}}}}
\newcommand{\ErrorTok}[1]{\textcolor[rgb]{0.64,0.00,0.00}{\textbf{#1}}}
\newcommand{\ExtensionTok}[1]{#1}
\newcommand{\FloatTok}[1]{\textcolor[rgb]{0.00,0.00,0.81}{#1}}
\newcommand{\FunctionTok}[1]{\textcolor[rgb]{0.00,0.00,0.00}{#1}}
\newcommand{\ImportTok}[1]{#1}
\newcommand{\InformationTok}[1]{\textcolor[rgb]{0.56,0.35,0.01}{\textbf{\textit{#1}}}}
\newcommand{\KeywordTok}[1]{\textcolor[rgb]{0.13,0.29,0.53}{\textbf{#1}}}
\newcommand{\NormalTok}[1]{#1}
\newcommand{\OperatorTok}[1]{\textcolor[rgb]{0.81,0.36,0.00}{\textbf{#1}}}
\newcommand{\OtherTok}[1]{\textcolor[rgb]{0.56,0.35,0.01}{#1}}
\newcommand{\PreprocessorTok}[1]{\textcolor[rgb]{0.56,0.35,0.01}{\textit{#1}}}
\newcommand{\RegionMarkerTok}[1]{#1}
\newcommand{\SpecialCharTok}[1]{\textcolor[rgb]{0.00,0.00,0.00}{#1}}
\newcommand{\SpecialStringTok}[1]{\textcolor[rgb]{0.31,0.60,0.02}{#1}}
\newcommand{\StringTok}[1]{\textcolor[rgb]{0.31,0.60,0.02}{#1}}
\newcommand{\VariableTok}[1]{\textcolor[rgb]{0.00,0.00,0.00}{#1}}
\newcommand{\VerbatimStringTok}[1]{\textcolor[rgb]{0.31,0.60,0.02}{#1}}
\newcommand{\WarningTok}[1]{\textcolor[rgb]{0.56,0.35,0.01}{\textbf{\textit{#1}}}}
\usepackage{longtable,booktabs,array}
\usepackage{calc} % for calculating minipage widths
% Correct order of tables after \paragraph or \subparagraph
\usepackage{etoolbox}
\makeatletter
\patchcmd\longtable{\par}{\if@noskipsec\mbox{}\fi\par}{}{}
\makeatother
% Allow footnotes in longtable head/foot
\IfFileExists{footnotehyper.sty}{\usepackage{footnotehyper}}{\usepackage{footnote}}
\makesavenoteenv{longtable}
\usepackage{graphicx}
\makeatletter
\def\maxwidth{\ifdim\Gin@nat@width>\linewidth\linewidth\else\Gin@nat@width\fi}
\def\maxheight{\ifdim\Gin@nat@height>\textheight\textheight\else\Gin@nat@height\fi}
\makeatother
% Scale images if necessary, so that they will not overflow the page
% margins by default, and it is still possible to overwrite the defaults
% using explicit options in \includegraphics[width, height, ...]{}
\setkeys{Gin}{width=\maxwidth,height=\maxheight,keepaspectratio}
% Set default figure placement to htbp
\makeatletter
\def\fps@figure{htbp}
\makeatother
\setlength{\emergencystretch}{3em} % prevent overfull lines
\providecommand{\tightlist}{%
  \setlength{\itemsep}{0pt}\setlength{\parskip}{0pt}}
\setcounter{secnumdepth}{-\maxdimen} % remove section numbering
\ifLuaTeX
  \usepackage{selnolig}  % disable illegal ligatures
\fi
\IfFileExists{bookmark.sty}{\usepackage{bookmark}}{\usepackage{hyperref}}
\IfFileExists{xurl.sty}{\usepackage{xurl}}{} % add URL line breaks if available
\urlstyle{same} % disable monospaced font for URLs
\hypersetup{
  pdftitle={Lista Entrega 5},
  pdfauthor={Davi Wentrick Feijó - 200016806},
  hidelinks,
  pdfcreator={LaTeX via pandoc}}

\title{Lista Entrega 5}
\author{Davi Wentrick Feijó - 200016806}
\date{2023-06-05}

\begin{document}
\maketitle

\hypertarget{questao-1-9.1}{%
\subsubsection{Questao 1 (9.1)}\label{questao-1-9.1}}

A questao nos dá a matriz de covariancia \(\rho\) e a matriz de erros
\(\Psi\)

A matriz \(\rho\)

\begin{longtable}[]{@{}rrr@{}}
\toprule()
\endhead
1.00 & 0.63 & 0.45 \\
0.63 & 1.00 & 0.35 \\
0.45 & 0.35 & 1.00 \\
\bottomrule()
\end{longtable}

A matriz \(\Psi\)

\begin{longtable}[]{@{}rrr@{}}
\toprule()
\endhead
0.19 & 0.00 & 0.00 \\
0.00 & 0.51 & 0.00 \\
0.00 & 0.00 & 0.75 \\
\bottomrule()
\end{longtable}

Sabemos que na analise fatorial temos a seguinte relacao:

\[
\Sigma = LL^T + \Psi
\] \[
LL^T = \Sigma - \Psi
\] Calculando \(LL^T\)

\begin{longtable}[]{@{}rrr@{}}
\toprule()
\endhead
0.81 & 0.63 & 0.45 \\
0.63 & 0.49 & 0.35 \\
0.45 & 0.35 & 0.25 \\
\bottomrule()
\end{longtable}

Podemos encontrar a comunalidade na diagonal da matriz \(LL^T\) já que
subtraimos o \(\Psi\)

\begin{longtable}[]{@{}r@{}}
\toprule()
x \\
\midrule()
\endhead
0.81 \\
0.49 \\
0.25 \\
\bottomrule()
\end{longtable}

Com essas informacoes podemos escrever nossa matriz \(\Sigma\) como:

\[
\Sigma = LL^T + \Psi
\]

\begin{Shaded}
\begin{Highlighting}[]
\NormalTok{p\_construido }\OtherTok{=}\NormalTok{ LLT }\SpecialCharTok{+}\NormalTok{ psi}
\end{Highlighting}
\end{Shaded}

\begin{longtable}[]{@{}rrr@{}}
\toprule()
\endhead
1.00 & 0.63 & 0.45 \\
0.63 & 1.00 & 0.35 \\
0.45 & 0.35 & 1.00 \\
\bottomrule()
\end{longtable}

\hypertarget{questao-2-9.2}{%
\subsubsection{Questao 2 (9.2)}\label{questao-2-9.2}}

\hypertarget{a}{%
\paragraph{A)}\label{a}}

As comunalidades sao:

\begin{Shaded}
\begin{Highlighting}[]
\NormalTok{comu}
\end{Highlighting}
\end{Shaded}

\begin{verbatim}
## [1] 0.81 0.49 0.25
\end{verbatim}

Podemos perceber que F1 detem a maior comunalidade logo é o fator que
mais explica a variancia dos dados

\hypertarget{b}{%
\paragraph{B)}\label{b}}

Sabemos que: \[
Cor(X,Y) = \frac{Cov(X,Y)}{S_xS_y}
\] \[
Cov(X,F) = L
\] Logo

\[
Cor(X_i,F_i) = \frac{Cov(X_i,F_i)}{S_iS_f} = \frac{L_i}{S_xS_f}
\]

\begin{Shaded}
\begin{Highlighting}[]
\NormalTok{cor\_xf }\OtherTok{=}\NormalTok{ Lestimado[}\DecValTok{1}\NormalTok{]}\SpecialCharTok{/}\NormalTok{(}\DecValTok{1}\SpecialCharTok{*}\NormalTok{comu[}\DecValTok{1}\NormalTok{])}
\end{Highlighting}
\end{Shaded}

\begin{verbatim}
## [1] -1.141896
\end{verbatim}

\newpage

\hypertarget{questao-3-9.3}{%
\subsubsection{Questao 3 (9.3)}\label{questao-3-9.3}}

\hypertarget{a-1}{%
\paragraph{A)}\label{a-1}}

Para realizar por meio de componentes principais primeiro precisamos
encontrar os autovalores e autovetores da matriz de correlacao aplicando
a decompisicao espectral em \(\rho\) dada na questao 9.1

\[
\rho = CDC^T
\]

\begin{Shaded}
\begin{Highlighting}[]
\NormalTok{eigen\_p }\OtherTok{=} \FunctionTok{eigen}\NormalTok{(p)}

\NormalTok{autoval }\OtherTok{\textless{}{-}}\NormalTok{ eigen\_p}\SpecialCharTok{$}\NormalTok{values}

\NormalTok{autovet }\OtherTok{\textless{}{-}}\NormalTok{ eigen\_p}\SpecialCharTok{$}\NormalTok{vectors}

\NormalTok{D }\OtherTok{\textless{}{-}} \FunctionTok{matrix}\NormalTok{(}\DecValTok{0}\NormalTok{, }\AttributeTok{nrow =} \DecValTok{3}\NormalTok{, }\AttributeTok{ncol =} \DecValTok{3}\NormalTok{)}
\FunctionTok{diag}\NormalTok{(D) }\OtherTok{\textless{}{-}} \FunctionTok{sqrt}\NormalTok{(autoval)}
\end{Highlighting}
\end{Shaded}

\begin{Shaded}
\begin{Highlighting}[]
\NormalTok{autoval}
\end{Highlighting}
\end{Shaded}

\begin{verbatim}
## [1] 1.9632830 0.6794930 0.3572239
\end{verbatim}

\begin{Shaded}
\begin{Highlighting}[]
\NormalTok{autovet}
\end{Highlighting}
\end{Shaded}

\begin{verbatim}
##            [,1]       [,2]       [,3]
## [1,] -0.6250027  0.2186276  0.7493822
## [2,] -0.5931510  0.4910833 -0.6379726
## [3,] -0.5074875 -0.8432314 -0.1772492
\end{verbatim}

Em seguida podemos encontrar nossa matriz L

\[
L = CD^{1/2}
\]

Aqui temos nossa matriz dos loadings

\begin{Shaded}
\begin{Highlighting}[]
\NormalTok{Lestimado}
\end{Highlighting}
\end{Shaded}

\begin{verbatim}
## [1] -0.8757363 -0.8311066 -0.7110772
\end{verbatim}

Para calcular a matriz \(\Psi\) temos que seguir a equacao:

\[
 \Psi = \Sigma - LL^T
\] Na diagonal obteremos nosso \(\Psi\)

\begin{Shaded}
\begin{Highlighting}[]
\NormalTok{psiestimado }\OtherTok{\textless{}{-}} \FunctionTok{diag}\NormalTok{(p}\SpecialCharTok{{-}}\NormalTok{LLT)}
\end{Highlighting}
\end{Shaded}

\begin{Shaded}
\begin{Highlighting}[]
\NormalTok{psiestimado}
\end{Highlighting}
\end{Shaded}

\begin{verbatim}
## [1] 0.2330860 0.3092618 0.4943692
\end{verbatim}

Para comparar com os resultados anteriores podemos aproximar a matrix
\(\Sigma\) de correlacoes por meio da formula:

\[
\Sigma = LL^T + \Psi
\]

\begin{verbatim}
##           [,1]      [,2]      [,3]
## [1,] 0.9569140 0.7278302 0.6227161
## [2,] 0.7278302 1.2007382 0.5909810
## [3,] 0.6227161 0.5909810 1.2556308
\end{verbatim}

\hypertarget{b-1}{%
\paragraph{B)}\label{b-1}}

A variancia explicada é:

\begin{verbatim}
## [1] 0.6544277 0.2264977 0.1190746
\end{verbatim}

Podemos notar que a primeira componente exxplica 65\% da variancia dos
dados

\newpage

\hypertarget{questao-4-9.19}{%
\subsubsection{Questao 4 (9.19)}\label{questao-4-9.19}}

\begin{longtable}[]{@{}
  >{\raggedright\arraybackslash}p{(\columnwidth - 14\tabcolsep) * \real{0.0411}}
  >{\raggedleft\arraybackslash}p{(\columnwidth - 14\tabcolsep) * \real{0.1370}}
  >{\raggedleft\arraybackslash}p{(\columnwidth - 14\tabcolsep) * \real{0.1370}}
  >{\raggedleft\arraybackslash}p{(\columnwidth - 14\tabcolsep) * \real{0.1370}}
  >{\raggedleft\arraybackslash}p{(\columnwidth - 14\tabcolsep) * \real{0.1370}}
  >{\raggedleft\arraybackslash}p{(\columnwidth - 14\tabcolsep) * \real{0.1370}}
  >{\raggedleft\arraybackslash}p{(\columnwidth - 14\tabcolsep) * \real{0.1370}}
  >{\raggedleft\arraybackslash}p{(\columnwidth - 14\tabcolsep) * \real{0.1370}}@{}}
\toprule()
\begin{minipage}[b]{\linewidth}\raggedright
\end{minipage} & \begin{minipage}[b]{\linewidth}\raggedleft
x1
\end{minipage} & \begin{minipage}[b]{\linewidth}\raggedleft
x2
\end{minipage} & \begin{minipage}[b]{\linewidth}\raggedleft
x3
\end{minipage} & \begin{minipage}[b]{\linewidth}\raggedleft
x4
\end{minipage} & \begin{minipage}[b]{\linewidth}\raggedleft
x5
\end{minipage} & \begin{minipage}[b]{\linewidth}\raggedleft
x6
\end{minipage} & \begin{minipage}[b]{\linewidth}\raggedleft
x7
\end{minipage} \\
\midrule()
\endhead
x1 & 1.0000000 & 0.9260758 & 0.8840023 & 0.5720363 & 0.7080738 &
0.6744073 & 0.9273116 \\
x2 & 0.9260758 & 1.0000000 & 0.8425232 & 0.5415080 & 0.7459097 &
0.4653880 & 0.9442960 \\
x3 & 0.8840023 & 0.8425232 & 1.0000000 & 0.7003630 & 0.6374712 &
0.6410886 & 0.8525682 \\
x4 & 0.5720363 & 0.5415080 & 0.7003630 & 1.0000000 & 0.5907360 &
0.1469074 & 0.4126395 \\
x5 & 0.7080738 & 0.7459097 & 0.6374712 & 0.5907360 & 1.0000000 &
0.3859502 & 0.5745533 \\
x6 & 0.6744073 & 0.4653880 & 0.6410886 & 0.1469074 & 0.3859502 &
1.0000000 & 0.5663721 \\
x7 & 0.9273116 & 0.9442960 & 0.8525682 & 0.4126395 & 0.5745533 &
0.5663721 & 1.0000000 \\
\bottomrule()
\end{longtable}

\newpage

\hypertarget{a-2}{%
\paragraph{A)}\label{a-2}}

Vamos usar a função principa() para obter a analise fatorial com m=2 e
m=3

m=2

\begin{Shaded}
\begin{Highlighting}[]
\NormalTok{AF2}
\end{Highlighting}
\end{Shaded}

\begin{verbatim}
## Principal Components Analysis
## Call: principal(r = cor_data, nfactors = 2, rotate = "none", n.obs = 50, 
##     covar = F)
## Standardized loadings (pattern matrix) based upon correlation matrix
##     PC1   PC2   h2    u2 com
## x1 0.97 -0.11 0.96 0.041 1.0
## x2 0.94  0.03 0.89 0.110 1.0
## x3 0.94  0.01 0.89 0.107 1.0
## x4 0.66  0.65 0.85 0.147 2.0
## x5 0.78  0.28 0.69 0.305 1.3
## x6 0.65 -0.62 0.81 0.194 2.0
## x7 0.91 -0.19 0.87 0.127 1.1
## 
##                        PC1  PC2
## SS loadings           5.03 0.93
## Proportion Var        0.72 0.13
## Cumulative Var        0.72 0.85
## Proportion Explained  0.84 0.16
## Cumulative Proportion 0.84 1.00
## 
## Mean item complexity =  1.3
## Test of the hypothesis that 2 components are sufficient.
## 
## The root mean square of the residuals (RMSR) is  0.08 
##  with the empirical chi square  11.93  with prob <  0.15 
## 
## Fit based upon off diagonal values = 0.99
\end{verbatim}

\newpage

m=3

\begin{Shaded}
\begin{Highlighting}[]
\NormalTok{AF3}
\end{Highlighting}
\end{Shaded}

\begin{verbatim}
## Principal Components Analysis
## Call: principal(r = cor_data, nfactors = 3, rotate = "none", n.obs = 50, 
##     covar = F)
## Standardized loadings (pattern matrix) based upon correlation matrix
##     PC1   PC2   PC3   h2    u2 com
## x1 0.97 -0.11 -0.05 0.96 0.039 1.0
## x2 0.94  0.03 -0.31 0.99 0.013 1.2
## x3 0.94  0.01  0.14 0.91 0.087 1.0
## x4 0.66  0.65  0.32 0.95 0.045 2.4
## x5 0.78  0.28  0.00 0.69 0.305 1.3
## x6 0.65 -0.62  0.43 0.99 0.012 2.7
## x7 0.91 -0.19 -0.31 0.97 0.033 1.3
## 
##                        PC1  PC2  PC3
## SS loadings           5.03 0.93 0.50
## Proportion Var        0.72 0.13 0.07
## Cumulative Var        0.72 0.85 0.92
## Proportion Explained  0.78 0.14 0.08
## Cumulative Proportion 0.78 0.92 1.00
## 
## Mean item complexity =  1.6
## Test of the hypothesis that 3 components are sufficient.
## 
## The root mean square of the residuals (RMSR) is  0.04 
##  with the empirical chi square  3.95  with prob <  0.27 
## 
## Fit based upon off diagonal values = 1
\end{verbatim}

\newpage

\hypertarget{b-2}{%
\paragraph{B)}\label{b-2}}

Aqui temos as mesmas analises porem rotacionadas com o metodos
``varimax''

\begin{Shaded}
\begin{Highlighting}[]
\NormalTok{AF2\_rotated}
\end{Highlighting}
\end{Shaded}

\begin{verbatim}
## Principal Components Analysis
## Call: principal(r = cor_data, nfactors = 2, rotate = "varimax", n.obs = 50, 
##     covar = F)
## Standardized loadings (pattern matrix) based upon correlation matrix
##     RC1   RC2   h2    u2 com
## x1 0.79  0.58 0.96 0.041 1.8
## x2 0.67  0.66 0.89 0.110 2.0
## x3 0.68  0.65 0.89 0.107 2.0
## x4 0.04  0.92 0.85 0.147 1.0
## x5 0.38  0.74 0.69 0.305 1.5
## x6 0.90 -0.01 0.81 0.194 1.0
## x7 0.80  0.48 0.87 0.127 1.6
## 
##                        RC1  RC2
## SS loadings           3.13 2.84
## Proportion Var        0.45 0.41
## Cumulative Var        0.45 0.85
## Proportion Explained  0.52 0.48
## Cumulative Proportion 0.52 1.00
## 
## Mean item complexity =  1.6
## Test of the hypothesis that 2 components are sufficient.
## 
## The root mean square of the residuals (RMSR) is  0.08 
##  with the empirical chi square  11.93  with prob <  0.15 
## 
## Fit based upon off diagonal values = 0.99
\end{verbatim}

\newpage

\begin{Shaded}
\begin{Highlighting}[]
\NormalTok{AF3\_rotated}
\end{Highlighting}
\end{Shaded}

\begin{verbatim}
## Principal Components Analysis
## Call: principal(r = cor_data, nfactors = 3, rotate = "varimax", n.obs = 50, 
##     covar = F)
## Standardized loadings (pattern matrix) based upon correlation matrix
##     RC1  RC2  RC3   h2    u2 com
## x1 0.78 0.39 0.45 0.96 0.039 2.1
## x2 0.91 0.36 0.19 0.99 0.013 1.4
## x3 0.62 0.55 0.48 0.91 0.087 2.9
## x4 0.21 0.95 0.05 0.95 0.045 1.1
## x5 0.55 0.61 0.15 0.69 0.305 2.1
## x6 0.29 0.06 0.95 0.99 0.012 1.2
## x7 0.91 0.18 0.33 0.97 0.033 1.3
## 
##                        RC1  RC2  RC3
## SS loadings           3.07 1.89 1.51
## Proportion Var        0.44 0.27 0.22
## Cumulative Var        0.44 0.71 0.92
## Proportion Explained  0.48 0.29 0.23
## Cumulative Proportion 0.48 0.77 1.00
## 
## Mean item complexity =  1.7
## Test of the hypothesis that 3 components are sufficient.
## 
## The root mean square of the residuals (RMSR) is  0.04 
##  with the empirical chi square  3.95  with prob <  0.27 
## 
## Fit based upon off diagonal values = 1
\end{verbatim}

\hypertarget{c}{%
\paragraph{C)}\label{c}}

Vamos obter as comunalidades, a variancia especifica e a matriz L sem a
rotação varimax

\begin{verbatim}
## A comunalidade com m=2 é:
##  0.95 0.89 0.89 0.44 0.61 0.42 0.84 0.01 0 0 0.42 0.08 0.39 0.04
\end{verbatim}

\begin{verbatim}
## A diagonal da matriz psi com m=2 é:
##  0.04 0.11 0.11 0.15 0.31 0.19 0.13
\end{verbatim}

\begin{verbatim}
## A matriz LLT:
\end{verbatim}

\begin{verbatim}
##       x1    x2    x3    x4    x5    x6    x7
## x1 0.959 0.914 0.918 0.573 0.731 0.698 0.910
## x2 0.914 0.890 0.891 0.641 0.747 0.594 0.856
## x3 0.918 0.891 0.893 0.630 0.743 0.607 0.862
## x4 0.573 0.641 0.630 0.853 0.701 0.028 0.479
## x5 0.731 0.747 0.743 0.701 0.695 0.331 0.661
## x6 0.698 0.594 0.607 0.028 0.331 0.806 0.713
## x7 0.910 0.856 0.862 0.479 0.661 0.713 0.873
\end{verbatim}

\begin{verbatim}
## Matriz de correlação aproximada:
\end{verbatim}

\begin{verbatim}
##        x1     x2     x3     x4     x5     x6     x7
## x1  0.000  0.012 -0.034 -0.001 -0.023 -0.024  0.017
## x2  0.012  0.000 -0.049 -0.099 -0.001 -0.129  0.088
## x3 -0.034 -0.049  0.000  0.071 -0.105  0.034 -0.009
## x4 -0.001 -0.099  0.071  0.000 -0.111  0.119 -0.066
## x5 -0.023 -0.001 -0.105 -0.111  0.000  0.055 -0.086
## x6 -0.024 -0.129  0.034  0.119  0.055  0.000 -0.147
## x7  0.017  0.088 -0.009 -0.066 -0.086 -0.147  0.000
\end{verbatim}

\begin{Shaded}
\begin{Highlighting}[]
\FunctionTok{cat}\NormalTok{(}\StringTok{"A comunalidade com m=2 é:}\SpecialCharTok{\textbackslash{}n}\StringTok{"}\NormalTok{,}\FunctionTok{round}\NormalTok{(AF3\_comu,}\DecValTok{2}\NormalTok{))}
\end{Highlighting}
\end{Shaded}

\begin{verbatim}
## A comunalidade com m=2 é:
##  0.92 0.84 0.84 0.29 0.48 0.27 0.76 0 0 0 0.27 0.02 -0.24 -0.01 0 -0.03 0 0.03 0 0.08 -0.03
\end{verbatim}

\begin{Shaded}
\begin{Highlighting}[]
\FunctionTok{cat}\NormalTok{(}\StringTok{"A diagonal da matriz psi com m=2 é:}\SpecialCharTok{\textbackslash{}n}\StringTok{"}\NormalTok{,}\FunctionTok{round}\NormalTok{(AF3}\SpecialCharTok{$}\NormalTok{uniquenesses,}\DecValTok{2}\NormalTok{))}
\end{Highlighting}
\end{Shaded}

\begin{verbatim}
## A diagonal da matriz psi com m=2 é:
##  0.04 0.01 0.09 0.05 0.31 0.01 0.03
\end{verbatim}

\begin{Shaded}
\begin{Highlighting}[]
\FunctionTok{cat}\NormalTok{(}\StringTok{"A matriz LLT:"}\NormalTok{)}
\end{Highlighting}
\end{Shaded}

\begin{verbatim}
## A matriz LLT:
\end{verbatim}

\begin{Shaded}
\begin{Highlighting}[]
\FunctionTok{round}\NormalTok{(AF3\_LLT,}\DecValTok{3}\NormalTok{)}
\end{Highlighting}
\end{Shaded}

\begin{verbatim}
##       x1    x2    x3    x4    x5    x6    x7
## x1 0.961 0.931 0.911 0.556 0.731 0.676 0.927
## x2 0.931 0.987 0.846 0.541 0.745 0.461 0.952
## x3 0.911 0.846 0.913 0.675 0.743 0.669 0.818
## x4 0.556 0.541 0.675 0.955 0.703 0.163 0.381
## x5 0.731 0.745 0.743 0.703 0.695 0.333 0.660
## x6 0.676 0.461 0.669 0.163 0.333 0.988 0.583
## x7 0.927 0.952 0.818 0.381 0.660 0.583 0.967
\end{verbatim}

\begin{Shaded}
\begin{Highlighting}[]
\FunctionTok{cat}\NormalTok{(}\StringTok{"Matriz de correlação aproximada:"}\NormalTok{)}
\end{Highlighting}
\end{Shaded}

\begin{verbatim}
## Matriz de correlação aproximada:
\end{verbatim}

\begin{Shaded}
\begin{Highlighting}[]
\NormalTok{resAF2}
\end{Highlighting}
\end{Shaded}

\begin{verbatim}
##        x1     x2     x3     x4     x5     x6     x7
## x1  0.000  0.012 -0.034 -0.001 -0.023 -0.024  0.017
## x2  0.012  0.000 -0.049 -0.099 -0.001 -0.129  0.088
## x3 -0.034 -0.049  0.000  0.071 -0.105  0.034 -0.009
## x4 -0.001 -0.099  0.071  0.000 -0.111  0.119 -0.066
## x5 -0.023 -0.001 -0.105 -0.111  0.000  0.055 -0.086
## x6 -0.024 -0.129  0.034  0.119  0.055  0.000 -0.147
## x7  0.017  0.088 -0.009 -0.066 -0.086 -0.147  0.000
\end{verbatim}

\hypertarget{d}{%
\paragraph{D)}\label{d}}

\[
\begin{itemize}
  \item Hipótese nula ($H_0$): ...
  \item Hipótese alternativa ($H_1$): ...
\end{itemize}
\]

\begin{Shaded}
\begin{Highlighting}[]
\CommentTok{\#d)}

\CommentTok{\#m=2}
\FunctionTok{dim}\NormalTok{(AF2}\SpecialCharTok{$}\NormalTok{loadings)}
\end{Highlighting}
\end{Shaded}

\begin{verbatim}
## [1] 7 2
\end{verbatim}

\begin{Shaded}
\begin{Highlighting}[]
\NormalTok{n }\OtherTok{=} \FunctionTok{dim}\NormalTok{(data)[}\DecValTok{1}\NormalTok{]}
\NormalTok{p }\OtherTok{=} \FunctionTok{dim}\NormalTok{(AF2}\SpecialCharTok{$}\NormalTok{loadings)[}\DecValTok{1}\NormalTok{]}
\NormalTok{m }\OtherTok{=} \FunctionTok{dim}\NormalTok{(AF2}\SpecialCharTok{$}\NormalTok{loadings)[}\DecValTok{2}\NormalTok{]}

\NormalTok{AF2\_teste\_stat }\OtherTok{=}\NormalTok{ AF2}\SpecialCharTok{$}\NormalTok{chi}
\NormalTok{AF2\_pvalue }\OtherTok{=}\NormalTok{ AF2}\SpecialCharTok{$}\NormalTok{PVAL}

\CommentTok{\#m=3}
\FunctionTok{dim}\NormalTok{(AF3}\SpecialCharTok{$}\NormalTok{loadings)}
\end{Highlighting}
\end{Shaded}

\begin{verbatim}
## [1] 7 3
\end{verbatim}

\begin{Shaded}
\begin{Highlighting}[]
\NormalTok{n }\OtherTok{=} \FunctionTok{dim}\NormalTok{(data)[}\DecValTok{1}\NormalTok{]}
\NormalTok{p }\OtherTok{=} \FunctionTok{dim}\NormalTok{(AF3}\SpecialCharTok{$}\NormalTok{loadings)[}\DecValTok{1}\NormalTok{]}
\NormalTok{m }\OtherTok{=} \FunctionTok{dim}\NormalTok{(AF3}\SpecialCharTok{$}\NormalTok{loadings)[}\DecValTok{2}\NormalTok{]}

\NormalTok{AF2\_teste\_stat }\OtherTok{=}\NormalTok{ AF3}\SpecialCharTok{$}\NormalTok{chi}
\NormalTok{AF2\_pvalue }\OtherTok{=}\NormalTok{ AF3}\SpecialCharTok{$}\NormalTok{PVAL}
\end{Highlighting}
\end{Shaded}


\end{document}
