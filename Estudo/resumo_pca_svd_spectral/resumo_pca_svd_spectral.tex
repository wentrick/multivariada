% Options for packages loaded elsewhere
\PassOptionsToPackage{unicode}{hyperref}
\PassOptionsToPackage{hyphens}{url}
%
\documentclass[
]{article}
\usepackage{amsmath,amssymb}
\usepackage{iftex}
\ifPDFTeX
  \usepackage[T1]{fontenc}
  \usepackage[utf8]{inputenc}
  \usepackage{textcomp} % provide euro and other symbols
\else % if luatex or xetex
  \usepackage{unicode-math} % this also loads fontspec
  \defaultfontfeatures{Scale=MatchLowercase}
  \defaultfontfeatures[\rmfamily]{Ligatures=TeX,Scale=1}
\fi
\usepackage{lmodern}
\ifPDFTeX\else
  % xetex/luatex font selection
\fi
% Use upquote if available, for straight quotes in verbatim environments
\IfFileExists{upquote.sty}{\usepackage{upquote}}{}
\IfFileExists{microtype.sty}{% use microtype if available
  \usepackage[]{microtype}
  \UseMicrotypeSet[protrusion]{basicmath} % disable protrusion for tt fonts
}{}
\makeatletter
\@ifundefined{KOMAClassName}{% if non-KOMA class
  \IfFileExists{parskip.sty}{%
    \usepackage{parskip}
  }{% else
    \setlength{\parindent}{0pt}
    \setlength{\parskip}{6pt plus 2pt minus 1pt}}
}{% if KOMA class
  \KOMAoptions{parskip=half}}
\makeatother
\usepackage{xcolor}
\usepackage[margin=1in]{geometry}
\usepackage{color}
\usepackage{fancyvrb}
\newcommand{\VerbBar}{|}
\newcommand{\VERB}{\Verb[commandchars=\\\{\}]}
\DefineVerbatimEnvironment{Highlighting}{Verbatim}{commandchars=\\\{\}}
% Add ',fontsize=\small' for more characters per line
\usepackage{framed}
\definecolor{shadecolor}{RGB}{248,248,248}
\newenvironment{Shaded}{\begin{snugshade}}{\end{snugshade}}
\newcommand{\AlertTok}[1]{\textcolor[rgb]{0.94,0.16,0.16}{#1}}
\newcommand{\AnnotationTok}[1]{\textcolor[rgb]{0.56,0.35,0.01}{\textbf{\textit{#1}}}}
\newcommand{\AttributeTok}[1]{\textcolor[rgb]{0.13,0.29,0.53}{#1}}
\newcommand{\BaseNTok}[1]{\textcolor[rgb]{0.00,0.00,0.81}{#1}}
\newcommand{\BuiltInTok}[1]{#1}
\newcommand{\CharTok}[1]{\textcolor[rgb]{0.31,0.60,0.02}{#1}}
\newcommand{\CommentTok}[1]{\textcolor[rgb]{0.56,0.35,0.01}{\textit{#1}}}
\newcommand{\CommentVarTok}[1]{\textcolor[rgb]{0.56,0.35,0.01}{\textbf{\textit{#1}}}}
\newcommand{\ConstantTok}[1]{\textcolor[rgb]{0.56,0.35,0.01}{#1}}
\newcommand{\ControlFlowTok}[1]{\textcolor[rgb]{0.13,0.29,0.53}{\textbf{#1}}}
\newcommand{\DataTypeTok}[1]{\textcolor[rgb]{0.13,0.29,0.53}{#1}}
\newcommand{\DecValTok}[1]{\textcolor[rgb]{0.00,0.00,0.81}{#1}}
\newcommand{\DocumentationTok}[1]{\textcolor[rgb]{0.56,0.35,0.01}{\textbf{\textit{#1}}}}
\newcommand{\ErrorTok}[1]{\textcolor[rgb]{0.64,0.00,0.00}{\textbf{#1}}}
\newcommand{\ExtensionTok}[1]{#1}
\newcommand{\FloatTok}[1]{\textcolor[rgb]{0.00,0.00,0.81}{#1}}
\newcommand{\FunctionTok}[1]{\textcolor[rgb]{0.13,0.29,0.53}{\textbf{#1}}}
\newcommand{\ImportTok}[1]{#1}
\newcommand{\InformationTok}[1]{\textcolor[rgb]{0.56,0.35,0.01}{\textbf{\textit{#1}}}}
\newcommand{\KeywordTok}[1]{\textcolor[rgb]{0.13,0.29,0.53}{\textbf{#1}}}
\newcommand{\NormalTok}[1]{#1}
\newcommand{\OperatorTok}[1]{\textcolor[rgb]{0.81,0.36,0.00}{\textbf{#1}}}
\newcommand{\OtherTok}[1]{\textcolor[rgb]{0.56,0.35,0.01}{#1}}
\newcommand{\PreprocessorTok}[1]{\textcolor[rgb]{0.56,0.35,0.01}{\textit{#1}}}
\newcommand{\RegionMarkerTok}[1]{#1}
\newcommand{\SpecialCharTok}[1]{\textcolor[rgb]{0.81,0.36,0.00}{\textbf{#1}}}
\newcommand{\SpecialStringTok}[1]{\textcolor[rgb]{0.31,0.60,0.02}{#1}}
\newcommand{\StringTok}[1]{\textcolor[rgb]{0.31,0.60,0.02}{#1}}
\newcommand{\VariableTok}[1]{\textcolor[rgb]{0.00,0.00,0.00}{#1}}
\newcommand{\VerbatimStringTok}[1]{\textcolor[rgb]{0.31,0.60,0.02}{#1}}
\newcommand{\WarningTok}[1]{\textcolor[rgb]{0.56,0.35,0.01}{\textbf{\textit{#1}}}}
\usepackage{graphicx}
\makeatletter
\def\maxwidth{\ifdim\Gin@nat@width>\linewidth\linewidth\else\Gin@nat@width\fi}
\def\maxheight{\ifdim\Gin@nat@height>\textheight\textheight\else\Gin@nat@height\fi}
\makeatother
% Scale images if necessary, so that they will not overflow the page
% margins by default, and it is still possible to overwrite the defaults
% using explicit options in \includegraphics[width, height, ...]{}
\setkeys{Gin}{width=\maxwidth,height=\maxheight,keepaspectratio}
% Set default figure placement to htbp
\makeatletter
\def\fps@figure{htbp}
\makeatother
\setlength{\emergencystretch}{3em} % prevent overfull lines
\providecommand{\tightlist}{%
  \setlength{\itemsep}{0pt}\setlength{\parskip}{0pt}}
\setcounter{secnumdepth}{-\maxdimen} % remove section numbering
\usepackage{booktabs}
\usepackage{longtable}
\usepackage{array}
\usepackage{multirow}
\usepackage{wrapfig}
\usepackage{float}
\usepackage{colortbl}
\usepackage{pdflscape}
\usepackage{tabu}
\usepackage{threeparttable}
\usepackage{threeparttablex}
\usepackage[normalem]{ulem}
\usepackage{makecell}
\usepackage{xcolor}
\ifLuaTeX
  \usepackage{selnolig}  % disable illegal ligatures
\fi
\IfFileExists{bookmark.sty}{\usepackage{bookmark}}{\usepackage{hyperref}}
\IfFileExists{xurl.sty}{\usepackage{xurl}}{} % add URL line breaks if available
\urlstyle{same}
\hypersetup{
  pdftitle={Decomposição Espectral, SVD e PCA},
  pdfauthor={Davi Wentrick Feijó - 200016806},
  hidelinks,
  pdfcreator={LaTeX via pandoc}}

\title{Decomposição Espectral, SVD e PCA}
\author{Davi Wentrick Feijó - 200016806}
\date{2023-05-23}

\begin{document}
\maketitle

\hypertarget{decomposiuxe7uxe3o-espectral}{%
\subsubsection{Decomposição
Espectral}\label{decomposiuxe7uxe3o-espectral}}

A decomposição espectral consiste em decompor uma matriz A no produto de
3 matrizes da seguinte forma:

\[
A_{n \times n} = P_{n \times n} D_{n \times n} P_{n \times n}^{T}
\]

Onde P é uma matriz onde as cada coluna é composta por um autovetor de A
e D e uma matriz diagonal com os autovalores correspondentes a cada
autovetor de A.

Podemos realizar isso dentro do R por meio da função eigen(). Vamos
utilizar a seguinte matriz e usaremos a matriz de covariancia dela (com
isso garantimos que ela é positiva definida)

\begin{verbatim}
##           [,1]      [,2]     [,3]      [,4]
## [1,]  99.00000  61.33333 20.50000 -29.66667
## [2,]  61.33333  40.33333 20.16667 -12.00000
## [3,]  20.50000  20.16667 30.25000  24.16667
## [4,] -29.66667 -12.00000 24.16667  72.33333
\end{verbatim}

Aqui vamos fazer nossa decomposicao, o resultado saira como um objeto
onde \$vectors é nosso P e \$values é a matriz D (temos que usar a
função diag() para que tenhamos a matriz diagona pois o resultado sai em
um vetor)

\begin{Shaded}
\begin{Highlighting}[]
\NormalTok{spectral\_decomposition }\OtherTok{=} \FunctionTok{eigen}\NormalTok{(A)}
\end{Highlighting}
\end{Shaded}

\begin{Shaded}
\begin{Highlighting}[]
\NormalTok{P }\OtherTok{=}\NormalTok{ spectral\_decomposition}\SpecialCharTok{$}\NormalTok{vectors}
\end{Highlighting}
\end{Shaded}

\begin{verbatim}
##            [,1]      [,2]       [,3]        [,4]
## [1,] -0.7943006 0.1158358  0.4105382  0.43258175
## [2,] -0.4936169 0.1906772 -0.1244255 -0.83934669
## [3,] -0.1506798 0.5193732 -0.7771652  0.32180960
## [4,]  0.3205066 0.8249096  0.4604248 -0.06934517
\end{verbatim}

\begin{Shaded}
\begin{Highlighting}[]
\NormalTok{D }\OtherTok{=} \FunctionTok{diag}\NormalTok{(spectral\_decomposition}\SpecialCharTok{$}\NormalTok{values)}
\end{Highlighting}
\end{Shaded}

\begin{verbatim}
##          [,1]     [,2]     [,3]          [,4]
## [1,] 152.9751  0.00000 0.000000  0.000000e+00
## [2,]   0.0000 80.60931 0.000000  0.000000e+00
## [3,]   0.0000  0.00000 8.332241  0.000000e+00
## [4,]   0.0000  0.00000 0.000000 -1.030665e-14
\end{verbatim}

Podemos confimar a decomposição tentando obter a matriz A novamente

\begin{Shaded}
\begin{Highlighting}[]
\NormalTok{A\_recuperado }\OtherTok{=}\NormalTok{ P }\SpecialCharTok{\%*\%}\NormalTok{ D }\SpecialCharTok{\%*\%} \FunctionTok{solve}\NormalTok{(P)}
\end{Highlighting}
\end{Shaded}

\begin{verbatim}
##           [,1]      [,2]     [,3]      [,4]
## [1,]  99.00000  61.33333 20.50000 -29.66667
## [2,]  61.33333  40.33333 20.16667 -12.00000
## [3,]  20.50000  20.16667 30.25000  24.16667
## [4,] -29.66667 -12.00000 24.16667  72.33333
\end{verbatim}

Vale notar que quando A é simetrica \(P^{-1} = P^{T}\)

\begin{Shaded}
\begin{Highlighting}[]
\NormalTok{P\_transposto }\OtherTok{=} \FunctionTok{t}\NormalTok{(P)}
\end{Highlighting}
\end{Shaded}

\begin{verbatim}
##            [,1]       [,2]       [,3]        [,4]
## [1,] -0.7943006 -0.4936169 -0.1506798  0.32050662
## [2,]  0.1158358  0.1906772  0.5193732  0.82490957
## [3,]  0.4105382 -0.1244255 -0.7771652  0.46042476
## [4,]  0.4325818 -0.8393467  0.3218096 -0.06934517
\end{verbatim}

\begin{Shaded}
\begin{Highlighting}[]
\NormalTok{P\_inversa }\OtherTok{=} \FunctionTok{solve}\NormalTok{(P)}
\end{Highlighting}
\end{Shaded}

\begin{verbatim}
##            [,1]       [,2]       [,3]        [,4]
## [1,] -0.7943006 -0.4936169 -0.1506798  0.32050662
## [2,]  0.1158358  0.1906772  0.5193732  0.82490957
## [3,]  0.4105382 -0.1244255 -0.7771652  0.46042476
## [4,]  0.4325818 -0.8393467  0.3218096 -0.06934517
\end{verbatim}

\newpage

\hypertarget{single-value-decomposition-svd}{%
\subsubsection{Single Value Decomposition
(SVD)}\label{single-value-decomposition-svd}}

O SVD assim como a decomposição espectral busca representar uma matriz A
pelo produto de 3 matrizes da seguinte forma:

\[
A_{m \times n} = U_{m \times n} \Sigma_{n \times n} V^{T}_{n \times n}
\]

Onde U e V sao matrizes ortogonais e \(\Sigma\) (ou D ou S) é uma matriz
diagonal.O cálculo da SVD consiste em encontrar os autovalores e
autovetores de \(AA^T\) e \(A^TA\). Os autovetores de \(A^TA\) compõem
as colunas de V, os autovetores de \(AA^T\) compõem as colunas de U.
Além disso, os valores singulares em S (Matriz de valores singulares)
são raízes quadradas dos autovalores de \(AA^T\) ou \(A^TA\). Os valores
singulares são as entradas diagonais da matriz S e estão dispostos em
ordem decrescente. Os valores singulares são sempre números reais. Se a
matriz A for uma matriz real, então U e V também são reais. Resumindo:

\[
U = AA^T
\]

\[
V = A^TA
\] \[
S = \text{Raiz quadrada dos autovalores de U ou V} 
\] Vale notar que ao aplicar essas transformações nós estamos
rotacionando os dados (ortonormalidade) com U, escalando os vetores com
S e rotacionando novamente (voltando ao estado inicial) com \(V^T\).

\begin{Shaded}
\begin{Highlighting}[]
\NormalTok{sv\_decomp }\OtherTok{=} \FunctionTok{svd}\NormalTok{(A)}
\end{Highlighting}
\end{Shaded}

\begin{verbatim}
## $d
## [1] 1.529751e+02 8.060931e+01 8.332241e+00 1.091777e-14
## 
## $u
##            [,1]       [,2]       [,3]        [,4]
## [1,] -0.7943006 -0.1158358  0.4105382  0.43258175
## [2,] -0.4936169 -0.1906772 -0.1244255 -0.83934669
## [3,] -0.1506798 -0.5193732 -0.7771652  0.32180960
## [4,]  0.3205066 -0.8249096  0.4604248 -0.06934517
## 
## $v
##            [,1]       [,2]       [,3]        [,4]
## [1,] -0.7943006 -0.1158358  0.4105382 -0.43258175
## [2,] -0.4936169 -0.1906772 -0.1244255  0.83934669
## [3,] -0.1506798 -0.5193732 -0.7771652 -0.32180960
## [4,]  0.3205066 -0.8249096  0.4604248  0.06934517
\end{verbatim}

Podemos verificar o SVD tentando recuparar a matriz A

\begin{Shaded}
\begin{Highlighting}[]
\NormalTok{sv\_decomp}\SpecialCharTok{$}\NormalTok{u }\SpecialCharTok{\%*\%} \FunctionTok{diag}\NormalTok{(sv\_decomp}\SpecialCharTok{$}\NormalTok{d) }\SpecialCharTok{\%*\%} \FunctionTok{t}\NormalTok{(sv\_decomp}\SpecialCharTok{$}\NormalTok{v)}
\end{Highlighting}
\end{Shaded}

\begin{verbatim}
##           [,1]      [,2]     [,3]      [,4]
## [1,]  99.00000  61.33333 20.50000 -29.66667
## [2,]  61.33333  40.33333 20.16667 -12.00000
## [3,]  20.50000  20.16667 30.25000  24.16667
## [4,] -29.66667 -12.00000 24.16667  72.33333
\end{verbatim}

\hypertarget{diferenuxe7as-entre-a-decomposiuxe7uxe3o-espectral-e-o-svd}{%
\subsubsection{Diferenças entre a Decomposição espectral e o
SVD}\label{diferenuxe7as-entre-a-decomposiuxe7uxe3o-espectral-e-o-svd}}

Considere a decomposição em autovalores \(A=PDP^{-1}\) e a SVD
\(A=U \Sigma V^{-1}\). Algumas diferenças-chave são as seguintes:

\begin{itemize}
\item
  Os vetores na matriz de decomposição em autovalores P não são
  necessariamente ortogonais, então a mudança de base não é uma simples
  rotação. Por outro lado, os vetores nas matrizes U e V na SVD são
  ortonormais, então eles representam rotações (e possivelmente
  reflexões).
\item
  Na SVD, as matrizes não diagonais U e V não são necessariamente
  inversas uma da outra. Geralmente, elas não têm relação entre si. Na
  decomposição em autovalores, as matrizes não diagonais P e \(P^{-1}\)
  são inversas uma da outra.
\item
  Na SVD, as entradas na matriz diagonal \(\Sigma\) são todas números
  reais e não negativos. Na decomposição em autovalores, as entradas de
  D podem ser qualquer número complexo - negativo, positivo, imaginário,
  qualquer coisa.
\item
  A SVD sempre existe para qualquer tipo de matriz retangular ou
  quadrada, enquanto a decomposição em autovalores só existe para
  matrizes quadradas e, mesmo entre as matrizes quadradas, às vezes ela
  não existe.
\end{itemize}

\hypertarget{analise-de-componentens-principais-pca}{%
\subsubsection{Analise de componentens principais
(PCA)}\label{analise-de-componentens-principais-pca}}

Aplicar o SVD em uma matriz de Covariancia ou Correlação podemos retirar
informacoes importantes sobre um conjunto de dados.

Mas antes vamos entender o que ocorre com os dados ao obter essas
matrizes.

\begin{itemize}
\item
  A matriz de covariancia ela mantem a escala dos dados e não é
  centralizada.
\item
  A matriz de correlaçao ela normaliza os dados (divide pela variancia)
  ou seja ela controla a escala das variaveis deixando todas ``iguais''
\item
  A centralização dos dados pode ser feita, que nada mais é que subtrair
  a media de cada coluna, centralizando os dados em torno de 0 é
  recomendado ser feita.
\end{itemize}

A PCA pode ser feita tanto por Decomposição espectral ou SVD. A
diferenca é que o SVD funciona em matrizes nao quadradas como vimos
antes. Podemos comparar os resultados obtidos manualmente com as funcoes
ja implementadas.

Podemos mostrar que a matriz de covariancia A centralizada é:

\[
A_{n \times n} = \frac{X_{n \times m}  X_{n \times m}^{T}}{n-1}
\] Podemos aplicar a decomposicao espectral em X e susbtituir na formula
\[
X_{m \times n} = U_{m \times n} \Sigma_{n \times n} V^{T}_{n \times n}
\] \[
A_{n \times n} = \frac{V \Sigma U^{T} U \Sigma V^{T}}{n-1}
\] \[
A_{n \times n} = V\frac{\Sigma^2 }{n-1}V^{T}
\] Com isso sabemos a relacao entre os valores singulares (diagonal do
SVD) e os autovalores (Decomposicao espectral)

\[
\lambda = \frac{\Sigma^2 }{n-1}
\]

\begin{Shaded}
\begin{Highlighting}[]
\NormalTok{A }\OtherTok{=} \FunctionTok{scale}\NormalTok{(A,}\AttributeTok{center =}\NormalTok{ T)}
\NormalTok{n }\OtherTok{=} \FunctionTok{dim}\NormalTok{(A)[}\DecValTok{1}\NormalTok{]}
\CommentTok{\# Executando a PCA com prcomp()}
\NormalTok{resultado\_prcomp }\OtherTok{\textless{}{-}} \FunctionTok{prcomp}\NormalTok{(A)}

\CommentTok{\# Executando a decomposição SVD com svd()}
\NormalTok{resultado\_svd }\OtherTok{\textless{}{-}} \FunctionTok{svd}\NormalTok{(A)}

\CommentTok{\# Executando a decomposição espectral com eigen()}
\NormalTok{resultado\_eigen }\OtherTok{\textless{}{-}} \FunctionTok{eigen}\NormalTok{(A)}

\CommentTok{\# Obtendo os componentes principais e autovalores usando prcomp()}
\NormalTok{autovetores\_prcomp }\OtherTok{\textless{}{-}}\NormalTok{ resultado\_prcomp}\SpecialCharTok{$}\NormalTok{rotation}
\NormalTok{autovalores\_prcomp }\OtherTok{\textless{}{-}}\NormalTok{ resultado\_prcomp}\SpecialCharTok{$}\NormalTok{sdev}\SpecialCharTok{\^{}}\DecValTok{2}
\NormalTok{componentes\_principais\_prcomp }\OtherTok{=}\NormalTok{ resultado\_prcomp}\SpecialCharTok{$}\NormalTok{x}

\CommentTok{\# Obtendo os componentes principais e autovalores usando a decomposição SVD manual}
\NormalTok{autovetores\_svd }\OtherTok{\textless{}{-}}\NormalTok{ resultado\_svd}\SpecialCharTok{$}\NormalTok{v}
\NormalTok{autovalores\_svd }\OtherTok{\textless{}{-}}\NormalTok{ (resultado\_svd}\SpecialCharTok{$}\NormalTok{d}\SpecialCharTok{\^{}}\DecValTok{2}\SpecialCharTok{/}\NormalTok{(n}\DecValTok{{-}1}\NormalTok{))}
\NormalTok{componentes\_principais\_svd }\OtherTok{=}\NormalTok{ A }\SpecialCharTok{\%*\%}\NormalTok{ resultado\_svd}\SpecialCharTok{$}\NormalTok{v}

\CommentTok{\# Obtendo os componentes principais e autovalores usando a decomposição SVD manual}
\NormalTok{autovetores\_eigen }\OtherTok{\textless{}{-}}\NormalTok{ resultado\_eigen}\SpecialCharTok{$}\NormalTok{vectors}
\NormalTok{autovalores\_eigen }\OtherTok{\textless{}{-}}\NormalTok{ resultado\_eigen}\SpecialCharTok{$}\NormalTok{values}
\NormalTok{componentes\_principais\_eigen }\OtherTok{=}\NormalTok{ A }\SpecialCharTok{\%*\%}\NormalTok{ resultado\_eigen}\SpecialCharTok{$}\NormalTok{vectors}
\end{Highlighting}
\end{Shaded}

\begin{verbatim}
## [1] "Autovetores (Rotacao):"
\end{verbatim}

\begin{verbatim}
## Prcomp:
\end{verbatim}

\begin{verbatim}
##             PC1        PC2        PC3         PC4
## [1,] -0.5415312  0.1716512 0.47709478  0.67056717
## [2,] -0.5371016  0.2358534 0.33916720 -0.73543227
## [3,]  0.3601088  0.9318450 0.01401336  0.04231065
## [4,]  0.5372033 -0.2158100 0.81064777 -0.08768606
\end{verbatim}

\begin{verbatim}
## [1] "SVD Manual:"
\end{verbatim}

\begin{verbatim}
##            [,1]       [,2]       [,3]        [,4]
## [1,] -0.5415312  0.1716512 0.47709478  0.67056717
## [2,] -0.5371016  0.2358534 0.33916720 -0.73543227
## [3,]  0.3601088  0.9318450 0.01401336  0.04231065
## [4,]  0.5372033 -0.2158100 0.81064777 -0.08768606
\end{verbatim}

\begin{verbatim}
## [1] "Eigen:"
\end{verbatim}

\begin{verbatim}
##            [,1]       [,2]       [,3]        [,4]
## [1,] -0.5415312  0.1716512 0.47709478  0.67056717
## [2,] -0.5371016  0.2358534 0.33916720 -0.73543227
## [3,]  0.3601088  0.9318450 0.01401336  0.04231065
## [4,]  0.5372033 -0.2158100 0.81064777 -0.08768606
\end{verbatim}

\begin{verbatim}
## [1] "Autovalores:"
\end{verbatim}

\begin{verbatim}
## Prcomp: 3.336 0.653 0.01 0
\end{verbatim}

\begin{verbatim}
## SVD Manual: 3.336 0.653 0.01 0
\end{verbatim}

\begin{verbatim}
## Eigen: 3.037 1.156 0.012 0
\end{verbatim}

\begin{verbatim}
## [1] "Componentes Principais (Rotacao):"
\end{verbatim}

\begin{verbatim}
## Prcomp:
\end{verbatim}

\begin{verbatim}
##             PC1          PC2         PC3           PC4
## [1,] -1.9516975  0.003282231  0.10652345 -9.714451e-17
## [2,] -1.0359992 -0.423394122 -0.13012990  1.249001e-16
## [3,]  0.9178242  1.129994474 -0.02102752  6.938894e-18
## [4,]  2.0698724 -0.709882584  0.04463398  1.110223e-16
\end{verbatim}

\begin{verbatim}
## [1] "SVD Manual:"
\end{verbatim}

\begin{verbatim}
##            [,1]         [,2]        [,3]          [,4]
## [1,] -1.9516975  0.003282231  0.10652345 -3.469447e-16
## [2,] -1.0359992 -0.423394122 -0.13012990 -1.179612e-16
## [3,]  0.9178242  1.129994474 -0.02102752  1.700029e-16
## [4,]  2.0698724 -0.709882584  0.04463398  3.469447e-16
\end{verbatim}

\begin{verbatim}
## [1] "Eigen:"
\end{verbatim}

\begin{verbatim}
##            [,1]         [,2]        [,3]          [,4]
## [1,] -1.9516975  0.003282231  0.10652345 -3.469447e-16
## [2,] -1.0359992 -0.423394122 -0.13012990 -1.179612e-16
## [3,]  0.9178242  1.129994474 -0.02102752  1.700029e-16
## [4,]  2.0698724 -0.709882584  0.04463398  3.469447e-16
\end{verbatim}

\end{document}
